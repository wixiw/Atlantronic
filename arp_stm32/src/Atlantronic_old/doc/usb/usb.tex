\chapter{USB}

\section{Introduction}

Une grande partie des fonctions implémentées sont utilisables depuis l'usb (via l'interface en ligne de commande usb\_interface ou glplot par exemple). Il y a plusieurs objectifs :
\begin{itemize}
\item récupérer les log
\item afficher des informations sur ce qui se passe en temps réel (tracer des courbes de vitesses, afficher les points vus par les hokuyos ...)
\item être capable de faire des tests unitaires et des tests d'intégration sur le robot.
\item commander le robot à distance lors des tests pour tester une stratégie avant de la coder
\end{itemize}

\section{Interfaces}
La carte est en mode device et ne présentent qu'une seule interface.
Endpoints :
\begin{itemize}
\item 0 : contrôle
\item 1 : bulk in (carte -> pc)
\item 2 : bulk out (pc -> carte)
\end{itemize}

\section{Format des messages}

Les messages qui transitent sur les endpoints bulk sont composés d'une entête et des données :

\begin{itemize}
\item taille (16 bit)
\item type (16 bits)
\item data
\end{itemize}

\cline

Taille maximale d'un message :
\begin{itemize}
\item sens carte -> pc : 65536 octets
\item sens pc -> carte : 64 octets (limitation actuelle de l'implémentation "simple" 1 message = 1 trame usb).
\end{itemize}

\clearpage
